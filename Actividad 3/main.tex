\documentclass[notitlepage]{report}
\usepackage{graphicx}
\usepackage[left=1in, right=1in, top=1in, bottom=1in]{geometry}
\usepackage[spanish]{babel}
\selectlanguage{spanish}
\spanishdecimal{.}
\usepackage[utf8]{inputenc}
    
\usepackage{titling}
\usepackage{lipsum}
\title{Actividad 3: Pandas }
\author{Borbón Fragoso Julio Cèsar }
\date{11 de febrero del 2019}

\begin{document}

\maketitle

\section*{Comandos en Pandas}
Pandas es una libreria para facilitar el manejo de datos en el lenguaje Python, tiene varias funciones a continuación se describen algunas de estas: \\

{\bfseries pd.read\_csv} : Con esta función podemos leer un archivo en formato CSV (comma-separated values) \\
{\bfseries pd.head()}: Está función sirve para leer los primeros archivos de una tabla, en el parentesis se puede ingersar el número de archivos que deseas leer. \\
{\bfseries df.tail()}: Parecido al head pero muestra los últimos archivos
\\
{\bfseries df.dtypes}: Nos muestra como se han leido de manera automatica nuestras variables. 
\\
{\bfseries pd.to\_datetime}: Cambia una columna escogida a la variable de DATETIME para poder tratarla como tiempo 
\\ 
{\bfseries pd.options}: Podemos configurar la configuración de pandas, como por ejemplo cambiar el número predeterminado de columnas que se muestran, etc. 
\\
{\bfseries pd.to\_numeric}: Cambia una columna para que puede ser detectada como variable numérica. 
\\
{\bfseries df.mean()}: Nos muestra la media de las columnas
\\
{\bfseries df.std/median/max/min} : Nos muestra la varianza, la mediana, el máximo y el mínimo respectivamente.
\\
{\bfseries df.describe()}: Muestra un resumen estadistico con información varia
\\
{\bfseries df.count}: Nos muestra la cantidad de datos en una columna
\\
{\bfseries df.drop}: Deshecha una columna de los datos. 
\\
{\bfseries pd.dataframe}: Da  estructura de datos a nuestro archivo 
\\
{\bfseries df.resample}: Método de convenencia para transformación de frecuencia, solo se puede usar en variables relacionadas con la fecha. \\
{\bfseries}
\\
\section*{Preguntas}
Se incluyen las siguientes preguntas con un fragmento de codigo de como se resolvieron. 

{\bfseries ¿Cómo le podrás determinar cuáles son los meses más lluviosos?}
\begin{verbatim} 
#La precipitación esta sumamente relacionada con la lluvia así que asumiremos
#que toda la cantida de precipitación es debido únicamente a las lluvias
#Declarando una nueva variable df10, usamos el comando set index y resample para sacar 
#la media de cada mes 
df10 = df9.set_index('FECHA').resample('M')["PRECIP"].mean()
#Se muestra df10 que es una tabla de datos y con esto podemos ver cuales meses han tenido mayor 
#precipitación 
df10

\end{verbatim}
--- \\
--- \\
{\bfseries ¿Cuáles son los meses más fríos y cuáles son los más cálidos?}
\begin{verbatim}
#se declara una tabla df11 que tenga la media de las temperaturas
df11 = df9.set_index('FECHA').resample('M')["TMAX", "TMIN"].mean()
#Se lee la tabla y se checan los valores
df11
\end{verbatim}
\\
{\bfseries ¿Cuáles han sido años muy húmedos/secos?}
\begin{verbatim}
#De la misma manera que la primer pregunta, pero cambiamos
#la frecuencia a anual. 
df12 = df9.set_index('FECHA').resample('Y')["PRECIP"].mean()
#Se imprime df12 para revisar sus datos
df12
\end{verbatim}
{\bfseries ¿Cuáles años han tenido inviernos fríos/calidos?}
\begin{verbatim}
#DE la misma manera que la pregunta anterior
df13 = df9.set_index('FECHA').resample('Y')["TMAX", "TMIN"].mean()
df13
\end{verbatim}
{\bfseries ¿Cómo ha venido siendo la temperatura mensual promedio en los últimos 20 años?}
\begin{verbatim}
#Se crea una nueva columna de la suma de los promedios
#de Tmin y Tmax y se divide entre 2 
df12['prom'] = (df12['TMAX'] + df12['TMIN'])/2
#Se muestran únicamente los últimos 20 datos
df12.tail(20)
\end{verbatim}
{\bfseries ¿Qué ha pasado con la precipitación en los últimos 20 años de datos?}
\begin{verbatim}
#Se modifica df10 para que tenga la media de precipitación anual
df10 = df9.set_index('FECHA').resample('Y')["PRECIP"].mean()
#Se muestran los últimos 20 datos
df10.tail(20)
\end{verbatim}

Con esto se da concluida la activdad 3 realizada para conocer mejor la libreria para Python, Pandas.
\end{document}
